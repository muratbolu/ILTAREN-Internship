\documentclass[12pt,a4paper]{article}
\usepackage{fullpage}
\usepackage{amsfonts, amsmath, pifont}
\usepackage{amsthm}
\usepackage{graphicx}
\usepackage{float}

\usepackage{tkz-euclide}
\usepackage{tikz}
\usepackage{pgfplots}
\pgfplotsset{compat=1.13}

\usepackage{inconsolata}
\usepackage{listings}
\usepackage{xcolor}
\usepackage[utf8]{inputenc}
\usepackage[T1]{fontenc}

\definecolor{codegreen}{rgb}{0,0.6,0}
\definecolor{codegray}{rgb}{0.5,0.5,0.5}
\definecolor{codepurple}{rgb}{0.58,0,0.82}
\definecolor{backcolour}{rgb}{0.95,0.95,0.92}
\lstdefinestyle{mystyle}{
    backgroundcolor=\color{backcolour},
    commentstyle=\color{codegreen},
    keywordstyle=\color{magenta},
    numberstyle=\tiny\color{codegray},
    stringstyle=\color{codepurple},
    basicstyle=\ttfamily\footnotesize,
    breakatwhitespace=false,
    breaklines=true,
    captionpos=b,
    keepspaces=true,
    numbers=left,
    numbersep=5pt,
    showspaces=false,
    showstringspaces=false,
    showtabs=false,
    tabsize=2
}
\lstset{style=myStyle}

% mathpazo must be above fontspec because we change the main font
\usepackage{mathpazo}
\usepackage{fontspec}
\usepackage{inconsolata}
\setmainfont{texgyrepagella}[
Extension      =.otf,
UprightFont    =*-regular,
BoldFont       =*-bold,
ItalicFont     =*-italic,
BoldItalicFont =*-bolditalic,
]
\setmonofont{inconsolata}
% Next two lines change \xi because it's really ugly in mathpazo!
\DeclareSymbolFont{CMletters}{OML}{cmm}{m}{it}
\DeclareMathSymbol{\xi}{\mathord}{CMletters}{"18}

\usepackage{indentfirst}

\title{Internship Report}
\author{Murat Bolu}

\begin{document}
\maketitle

My internship at TÜBİTAK BİLGEM İLTAREN was from 1--29 July 2024 at the Ümitköy
campus. My department was system administration software, and I completed the
tasks in C++. My progress was monitored primarily by Emre Balcı and Murat
Özdemiray.

\section{Static Data Structures}

I started the internship writing statically allocated data structures from
scratch. Even though the C++ standard library provides some statically allocated
data structures like \texttt{std::array}, I was requested to write them by
myself and avoid using the standard library. Statically-allocated data
structures are preferred in low-level applications for several reasons. First,
they do not use \texttt{malloc} or \texttt{new}. These methods require an
operating system or an equivalent program that will provide the necessary
memory. These kinds of memory managers are not present in embedded applications
and must be avoided. Second, they do not utilize the stack. While the stack may
feel like an infinite memory source, stack overflow poses a significant risk for
such applications. Statically allocated data structures allocate their memory at
once while the program starts, and the stack is not utilized except for small
memory allocations like \texttt{int}'s and pointers.

\section{Traveling cities problem}

\subsection{Province distance data}

In the first internship problem, I was provided an XLS file containing the
distances between each pair of provinces in Turkey. I exported that XLS file to
a CSV file and did some very light preprocessing by hand. I removed the first
two lines containing the title text and junk information. Only 81 lines
containing the names of the provinces and the distances from that province,
separated by semicolons, were left. I wrote a parser that imported that text
file into a data structure in memory. Finally, I had an array of arrays, with
size \(81 \times 81\), containing the distances as a number.

\subsection{Problem description}

With this data, my task was to visit as many provinces as possible, given some
constraints. The distances between provinces were not considered except when
applying constraints. After the constraints were applied, the problem
transformed into the longest simple path problem in an undirected, unweighted
graph, with the goal of visiting most vertices.

The longest simple path problem is finding the longest path in a graph without
visiting any vertex more than once. It is an NP-Hard problem, and even the
approximations are computationally complex. The best way to approach this
problem is through heuristics. I have experimented with several heuristics and
chosen to implement some of them.

In our graph, all pairs of provinces are connected through an edge. These edges
have a value corresponding to the distance. Naturally, some of these edges must
be invalid since the longest simple path in a connected graph is trivial. Such
edges are invalidated through three constraints. The first is the starting
province, and the second and the third are \(X\) and \(Y\) values. \(X\) value
represents the distance, and \(Y\) value is the error margin of that distance.
For example, if \(X\) is 200 and \(Y\) is 50, only edges with values between 150
and 250 are valid.

As mentioned, when the constraints are applied, the problem turns into the
longest simple path in an undirected, unweighted graph, starting from a
particular vertex. To tackle this problem, I first implemented the brute-force
algorithm. It tried to solve for the longest route, na{\"\i}vely searching in
the space of routes. I added printing so that when it found a longer route than
the previous maximum, it printed the length of the route and the route itself.
The brute-force algorithm wasn't efficient because there were many redundancies
while searching. I looked for heuristics to improve my search algorithm.

\subsection{Heuristics research}

I researched longest path heuristics and started reading the ``Solving the
Longest Simple Path Problem with Heuristic Search'' paper by Yossi Cohen, Roni
Stern, and Ariel Felner. It gave me some ideas, like caching the previous states
and utilizing bi-connected components.

\subsubsection{Caching}

I implemented caching within the context of my problem's requirements. The
algorithm searches the space of states and caches the best state seen so far.
States hold the visited provinces in order, in a stack structure. States also
keep a boolean array of visited provinces and the number of visited provinces
for efficiency purposes. The best state is the state with the most visited
provinces. The caching works as follows. When the algorithm encounters a new
state, it compares its visited boolean array with the best state's visited
boolean array. The state is skipped if the former is a subset of the latter. The
reason is that if the best state contains all of the provinces visited by the
current state, that branch is explored and can be skipped.

\subsubsection{Bi-connected components}

I tried implementing the bi-connected components heuristic. However, it didn't
work as well as I expected. Even though I decided to scrap it, I'll explain it
briefly here. The main idea is that a graph can be divided into bi-connected
components, called block-cutpoint-tree. These components are only connected by a
single edge, meaning that if a simple path is being searched, one cannot ``go
back'' to a bi-connected component again. So, fully exploring a bi-connected
component is the best idea before going to the next component.
Block-cutpoint-tree can be constructed in linear time, so it isn't
computationally costly.

\subsubsection{Other attempts}

I also tried to implement other algorithms, like Kahn's algorithm for
topological sorting and Tarjan's strongly connected components algorithm. These
didn't work either because we could not consider our graph to be directed. In
hindsight, it was intuitive that these approaches didn't work. If they worked,
they would have been working in linear time, directly contradicting the
NP-Hardness of the problem, assuming \(P \neq\ NP\).

\subsection{Longest route with best \(X\) and \(Y\) values}

My last task for this problem was finding the longest route with the best \(X\)
and \(Y\) values. The best can be defined arbitrarily, and I was requested to
minimize the \(X + Y^2\) value. I also fixed the starting province to Ankara for
this task. The algorithm searches starting from the smallest \(X + Y^2\) values,
trying to find the longest path starting in Ankara. Since my algorithm takes too
long to find the longest route, I applied a heuristic for early termination. If
after a set number of iterations, say, ten thousand, the length of the best
route does not change, the algorithm exits. Early termination allows the
algorithm to exit if it cannot find a better route.

\section{Bus scheduling problem}

\subsection{Problem description}

\subsubsection{Generation}

In the second internship task, I had to generate and analyze the data myself.
Data generation is trivial; it is the timetable of a bus stop. Buses arrive at
the bus stop intermittently, with fixed periods. The timetable includes the
hours and minutes, and the number of buses that came at said times. Some buses
have a single period and arrive at the stop at every period. Some buses have
varying periods and arrive at the stop every other period. For example, if a bus
has a period of 10 minutes, it will arrive at the stop every 10 minutes and will
increase the counter for that timestamp by one. If a bus has alternating periods
of 5 and 7 minutes, it will arrive at \(5^\text{th}\), \(12^\text{th}\), and
\(17^\text{th}\) minute marks until the ending time. It should be noted that
only the number of buses that arrived at a specific time is known.

\subsubsection{Analysis}

The analysis of data is the trickier part. It aims to detangle the buses and
identify them correctly. The problem would have been significantly more
straightforward if the buses only had fixed periods. The algorithm would
identify the bus with the highest frequency by observing the first data point
and ``subtract'' its effect from the timetable. Then, it would identify the bus
with the subsequent highest frequency using the same approach. Indeed, I applied
this approach first, and it worked sufficiently for the trivial case. However,
alternating periods pose a problem in this case. Even though two alternating
periods have a proper period, which is their sum, the algorithm cannot work by
trying to find the highest frequency by observing the first data point. I had to
have a different approach.

\subsection{Algorithm}

My approach identifies the periods from the smallest to the largest and can
discern between alternating and constant periods. If the period is alternating,
the algorithm determines the sum of two oscillating periods and tries to
identify the two different components. If the period is constant, it finds two
identical ``oscillating'' components and reports one of them as the constant
period instead.

There are two caveats to my approach. The first is that the data generator only
generates periods smaller than half of the total time, i.e., it only identifies
periods with at least two instances. This is known in signal processing theory
as the Nyquist theorem. If a bus with a large period only arrives at the stop
once, its period and type cannot be determined correctly. Second, while my
algorithm can identify all buses correctly for almost all test cases, it can
only identify some buses for extensive test cases. This issue may be because
some buses have the same alternating periods, which cannot be detangled
correctly. I tried to augment my algorithm to detect these cases individually.
However, I couldn't do it correctly.

\subsection{Fourier Transform approach}

I also tried to utilize some of the techniques I learned in my signal processing
course at university. I implemented the fast Fourier transform algorithm and
analyzed the Fourier transform of the time series data. Theoretically, Fourier
transform could identify the data's periodicity and yield the buses' periods.
However, it was unnecessary for constant periods because the simple approach
worked fine. For the alternating periods, I couldn't manage to analyze the FT
data to extract the periods. Additionally, with the FT approach, the second
caveat I mentioned earlier, identifying the buses with the same total
alternating periods, would have persisted. That's why I abandoned this idea and
implemented an ad-hoc algorithm.

%\lstinputlisting[language=C++, caption=\texttt{file.cpp}]{../src/analyzer.cpp}

\end{document}
